\documentclass[11pt]{amsart}
\usepackage{geometry}                % See geometry.pdf to learn the layout options. There are lots.
\geometry{letterpaper}                   % ... or a4paper or a5paper or ... 
%\geometry{landscape}                % Activate for for rotated page geometry
%\usepackage[parfill]{parskip}    % Activate to begin paragraphs with an empty line rather than an indent
\usepackage{graphicx}
\usepackage{amssymb}
\usepackage{epstopdf}
\DeclareGraphicsRule{.tif}{png}{.png}{`convert #1 `dirname #1`/`basename #1 .tif`.png}

\title{Brief Article}
\author{The Author}
%\date{}                                           % Activate to display a given date or no date

\begin{document}
\maketitle
%\section{}
%\subsection{}

%https://en.wikipedia.org/wiki/Condition_number
%Thus, if the condition number is large, even a small error in b may cause a large error in x. On the other hand, if the condition number is small then the error in x will not be much bigger than the error in b.

A computer can rarely compute the exact solution to a system of linear equations due to the round off error inherit in floating point arithmetic.
The condition number of the coefficient matrix plays a central role in the error analysis of the computed solution.

\end{document}  