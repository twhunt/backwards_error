\documentclass[11pt]{amsart}
\usepackage{geometry}                % See geometry.pdf to learn the layout options. There are lots.
\geometry{letterpaper}                   % ... or a4paper or a5paper or ... 
%\geometry{landscape}                % Activate for for rotated page geometry
%\usepackage[parfill]{parskip}    % Activate to begin paragraphs with an empty line rather than an indent
\usepackage{graphicx}
\usepackage{amssymb}
\usepackage{epstopdf}
\DeclareGraphicsRule{.tif}{png}{.png}{`convert #1 `dirname #1`/`basename #1 .tif`.png}

\title{Brief Article}
\author{The Author}
%\date{}                                           % Activate to display a given date or no date

\begin{document}
\maketitle
%\section{}
%\subsection{}

%https://en.wikipedia.org/wiki/Condition_number
%Thus, if the condition number is large, even a small error in b may cause a large error in x. On the other hand, if the condition number is small then the error in x will not be much bigger than the error in b.

The solution to a system of linear equations computed by an algorithm on a computer is rarely exact, which prompts the question of how much error should one expect to creep into the answer returned by the algorithm?
Any one of the standard texts on numerical linear algebra provides an answer to this question, in excruciatingly precise detail.
The two things I hope to convey in this blog post. 
The first is a high level summary of the type of error analysis you can find in a numerical linear algebra text, and the second is a demonstration of how a commonly held interpretation of the standard error analysis does not actually explain the source of error in a computed solution.
Apologies to those who have said all this before, probably much more clearly.

If we set out to solve \(Ax=b\), then we will have to employ a specific algorithm, and let's call the output of our chosen algorithm \(\hat{x}\).
The \emph{absolute error} of the solution is \(x-\hat{x}\), that is, the difference between the computed solution which we know since that's what the algorithm gave us, and the exact solution, which we don't know, but hope is close to the computed solution.

A computer rarely computes the exact solution to a system of linear equations due to the round off error inherit in floating point arithmetic.
The condition number of the coefficient matrix plays a central role in the error analysis of the computed solution, and
the error analysis for a particular algorithm is usually fairly involved.
Suppose we set out to solve the system \(A x = b\) with a particular algorithm, say Gaussian elimination with partial pivoting, and the algorithm returns the solution \(\hat{x}\).
Because of round off error, \(\hat{x}\) typically will differ from the true solution \(x\) by some amount, which prompts the question, by how much?

If we express the system in matrix notation as \(Ax=b\), then the following quote from the Wikipedia condition number article is a representative interpretation of how, ``Thus, if the condition number is large, even a small error in \(b\) may cause a large error in \(x\). On the other hand, if the condition number is small then the error in \(x\) will not be much bigger than the error in \(b\)."

\end{document}  